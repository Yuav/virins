	I dette kapitlet spesifiseres kravene til systemet som skal utvikles. Kravene er delt inn i funksjonelle og ikke-funksjonelle krav. Funksjonelle krav beskriver hva et system er i stand til � gj�re. De spesifiserer ikke kvaliteten, effektiviteten eller implementasjonen av disse funksjonene. Med andre ord, the sier \emph{hva} en bruker kan gj�re med systemet. Ikke-funksjonelle krav beskriver egenskaper ved systemet som ikke er direkte funksjoner. De er relatert til systemet som en helhet. Tabell \ref{table:f_req_spec} og \ref{table:nf_req_spec} viser de funksjonelle og ikke-funksjonelle kravene til systemet som skal lages.
	
	\begin{table}[h]
		\centering
		\begin{tabular}{| c | c | p{9 cm} |}
			\hline 
			\bf ID & \bf Prior. & \bf Description \\
			\hline
			\hline
			FR1 & H�y & Systemet skal spore posisjonen til et menneskehode i tre dimensjoner. \\
			FR2 & H�y & Den sporede posisjonen skal oppdateres n�r hodet beveger p� seg. \\
			FR3 & Lav & En tre-dimensjonal scene skal vises p� en skjerm foran menneskehodet hvor kameraposisjonen i scenen er lik posisjonen til hodet i rommet. \\
			\hline
		\end{tabular}
		\caption{Funksjonelle krav}
		\label{table:f_req_spec}
	\end{table}
	
	\begin{table}[h]
		\centering
		\begin{tabular}{| c | c | p{9 cm} |}
			\hline 
			\bf ID & \bf Prior. & \bf Description \\
			\hline
			\hline
			NFR1 & H�y & Prisen for systemet skal v�re under 10 000 NOK. \\
			NFR2 & Lav & Eventuelt utstyr brukeren m� ha p� seg skal v�re lettere enn 300 gram. \\
			\hline
		\end{tabular}
		\caption{Ikke-funksjonelle krav}
		\label{table:nf_req_spec}
	\end{table}