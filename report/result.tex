\section{Hva prosjektet endte opp med}

	Etter implementasjonen endte vi opp med en adaptiv l�sning. I tillegg til � kunne bruke HD-kameraet kan det brukes Nintendo Wiimotes. Systemet tilbyr virtuell virkelighet etter kravspesifikasjonen, og st�tter f�lgende konfigurasjoner:
	
	\begin{itemize}
		\item Et HD-kamera som filmer to lys p� hodet
		\item En Wiimote som filmer to lys p� hodet
		\item To Wiimotes som filmer ett lys p� hodet
	\end{itemize}
	
	Felles for alle er at et eller flere kamera (Wiimote eller HD-kamera) filmer lys montert p� hodet. Programvaren bruker disse bildene til � estimere posisjonen til hodet i rommet.
	
	\begin{description}
		\item[Et HD-kamera som filmer to lys p� hodet] Her er det et HD-kamera som filmer to infrar�de LEDs p� hodet til brukeren. I denne konfigurasjonen f�lger systemet designen beskrevet i designseksjonen over til punkt og prikke.
		\item[En Wiimote som filmer to lys p� hodet] I denne konfigurasjonen er HD-kameraet byttet ut med en Nintendo Wiimote som ogs� tar seg av � finne koordinatene til lyspunktene i bildet som filmes. Disse koordinatene sendes tr�dl�st til v�r programvare hvor de brukes til � beregne hodeposisjon og vise en 3D-scene akkurat som i konfigurasjonen over.
		\item[To Wiimotes som filmer ett lys p� hodet] I denne konfigurasjonen brukes {\em to} Wiimotes. Dette betyr at kun ett lys trengs p� hodet.
	\end{description}

\section{Evaluering av resultatet}

	%L�sningen f�lger kravspekken og designen
	%HD-kamera har latency
	%Headset er �ss
	%To Wiimotes er bra, rotasjonsinvariant, h�yere oppl�sning, etc
	
	L�sningen vi endte opp med f�lger kravspesifikasjonene og designen. Den estimerer posisjonen til hodet ved � filme infrar�de LEDs montert p� hodet, og bruker denne til � skape en virtuell virkelighet p� skjermen foran brukeren. Vi starter evaluering ved � p�peke svakhetene ved systemet, og avslutter med styrkene.
	
	\subsection{Svakheter ved systemet}
	
		\subsubsection{HD-kameraet har forsinkelse}
		
			
		
		\subsubsection{LED-headset er suboptimalt}
		
			
	
	\subsection{Styrker ved systemet}
	
		\subsubsection{HD-kamera har h�y oppl�sning}
		
			
		
		\subsubsection{St�tte for to Wiimotes gir fordeler}
		
			
		
		\subsubsection{Systemet er portabelt og fleksibelt}
		
			
		
		\subsubsection{Virkelighetsf�lelsen er sterk}
		
			

\section{Konklusjon}

	TODO: Hva er svaret p� problemstillingen?

\section{Videre arbeid}

	