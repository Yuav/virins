Faget Eksperter i Team (EiT) gjennomf�res for tiln�rmet samtlige mastergradstudenter ved NTNU. Det blir stadig mer vanlig � jobbe i team. Faget organiseres i landsbyer med et gitt tema, og i landsbyen v�r har alle grupper hatt forskjellige prosjekter utenom to grupper, oss og en annen gruppe. Men etter problemstilling og vinkling av oppgaven s� vi raskt at vi hadde valgt ulike veier. Faget skal trene studentene i tverrfaglig samarbeid og gi innsikt i egen adferd, samt hvordan den p�virker og lar seg p�virke av gruppen. L�ringsprosessen til Eksperter i Team er utradisjonell i forhold til andre fag ved universitetet. Fagl�rerne har en fasilitatorrolle, hvor det ikke bare er fokus p� det faglige, men ogs� arbeidsprosessen til gruppene. Vi har tolket budskapet til EiT som f�lger: 

\begin{itemize}
	\item � ha innsikt i egen faglig kompetanse, og hvordan dette kan brukes i prosjektarbeidet.
	\item � ha innsikt i hvordan egen adferd p�virker gruppen, og hvordan man selv blir p�virket av gruppen.
	\item � utvikle ferdigheter til � l�se en tverrfaglig problemstilling. � 
\end{itemize}
	
VR- landsbyen stiller relativt store krav til datakompetanse og dette har gitt oss en del utfordringer under arbeidet med oppgaven. Gruppen v�r valgte oppgaven "Inexpensive tracked IR". Oppgavevalget var hovedsaklig basert p� felles interesse blant medlemmene samtidig som vi vektla gruppemedlemmenes faglige bakgrunn. Det faktum at to i gruppa ikke hadde noen datakunnskaper p� forh�nd ble tatt til vurdering. Dermed var det viktig for oss � velge en oppgave der vi f�lte alle kunne bidra og lettere tilegne seg n�dvendig kunnskap for � kunne l�se oppgaven. N�rmere utforming av oppgaven har blitt gjort sammen med fagleder Egil Tj�land samt ansatte hos Statoil Hydro i Bergen. Oppgaven g�r kort fortalt ut p� � lage et billig men godt system for tracking til bruk i virtuell virkelighet. Med tracking menes � spore posisjonen, rotasjonen, farten og lignende til et objekt (f.eks. hodet). Potensielle bruksomr�der er simulering og spill.

Hver onsdag gruppen har arbeidet, har det blitt skrevet dagbok. I tillegg har viktige refleksjoner blitt notert ned, og blant annet arbeidsplanlegging og resultater fra personlighetstester har blitt vurdert underveis. Mot slutten av arbeidet har prosessen blitt vurdert mer grundig opp mot relevant teori, i et fors�k p� � se mer dyptliggende sammenhenger. Prosessrapporten har derfor gradvis blitt til, og det endelige produktet er ser vi p� som et l�pende produkt. 

Prosessrapporten er skrevet i fellesskap, det vil si at flere gruppemedlemmer har jobbet samtidig med rapporten. Den har nesten utelukkende blitt skrevet med minimum to medlemmer til stede, noe som har fungert bra som bidrag til diskusjon og vurdering. Rapporten kan derfor i h�yeste grad betraktes som et felles produkt, som alle har deltatt aktivt i.
