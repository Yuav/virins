\section{Kravspesifikasjon}
	
	Funksjonelle krav beskriver hva et system er i stand til \aa \space gj\o re. De spesifiserer ikke kvaliteten, effektiviteten eller implementasjonen av disse funksjonene. Med andre ord, the sier \emph{hva} en bruker kan gj\o re med systemet. Ikke-funksjonelle krav beskriver egenskaper ved systemet som ikke er direkte funksjoner. De er relatert til systemet som en helhet.
	
	Tabell \ref{table:freq_spec} viser de funksjonelle kravene til systemet som skal lages. Tabell \ref{table:nfreq_spec} viser de ikke-funksjonelle kravene.
	
	\begin{table}
		\centering
		\begin{tabular}{| c | c | p{8 cm} |}
			\hline 
			\bf ID & \bf Prioritering & \bf Description \\
			\hline
			\hline
			FR1 & H\o y & Systemet skal spore posisjonen til et menneskehode i tre dimensjoner. \\
			FR2 & H\o y & Den sporede posisjonen skal oppdateres n\aa r hodet beveger p\aa \space seg. \\
			FR3 & Lav & En tre-dimensjonal scene skal vises p\aa \space en skjerm foran menneskehodet hvor kameraposisjonen i scenen er lik posisjonen til hodet i rommet. \\
			\hline
		\end{tabular}
		\caption{Funksjonelle krav}
		\label{table:freq_spec}
	\end{table}
	
	\begin{table}
		\centering
		\begin{tabular}{| c | c | p{8 cm} |}
			\hline 
			\bf ID & \bf Prioritering & \bf Description \\
			\hline
			\hline
			NFR1 & H\o y & Prisen for systemet skal v\ae re under 10 000 NOK. \\
			NFR2 & Lav & Eventuelt utstyr brukeren m\aa \space ha p\aa \space seg skal v\ae re lettere enn 300 gram. \\
			\hline
		\end{tabular}
		\caption{Ikke-funksjonelle krav}
		\label{table:nfreq_spec}
	\end{table}