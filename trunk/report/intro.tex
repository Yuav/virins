\section{Ressurser}

	Dette er krevende prosjekt som trenger personer med gode faglige og teknologiske kunnskaper innen data, elektronikk og petroleum. Og med personer i gruppa innefor alle disse faglige feltene ble prosjektet gjennomf\o rbart. Vi fikk ogs\aa \space mye hjelp fra ressurssterke personer utenfor gruppa innen data og elektronikk og vi fikk god hjelp fra Omega-verksted og IPT verkstedet.
	
	Dette prosjektet er ganske omfattende og kostnadskrevende med innkj\o p av dyrt elektronisk utstyr. Derfor er samarbeidet og den \o konomiske st\o tten fra StatoilHydro sv\ae rt viktig. Det er blitt gjort mye innkj\o p av utstyr i forbindelse med prosjektet selv om ikke alt er brukt i v\aa r endelige prototyp. Institutt for Petroleumsteknologi har gitt oss tilgjenglighet til mye avansert datautstyr som har v\ae rt viktig for gjennomf\o relsen av prosjektet.


\section{Problemstilling}

	Hvordan f\aa \space lagd en rimelig og brukbar prototyp for tracking av hode til bruk i VR

	\begin{itemize}
		\item Lav kostnad i forhold til lignende brukter p\aa \space markedet i dag
		\item Enkelt oppsett og brukevennlig
	\end{itemize}
	
	
Oppgaven som gruppa valgte var Inexpensive Tracked VR. Dette var fordi gruppa synes denne oppgaven hadde h\o yeste wow faktor, og fordi gruppa f\o lte at kompentansen for \aa \space l\o se den var tilstede. Gruppa ble enige om \aa \space bruke et headset til \aa \space feste IR dioder p\aa , og s\aa \space tracke dem ved hjelp av et hd kamera for \aa \space kunne st\aa \space et stykke unna skjermen.
